\chapter{Retrospettiva}\label{ch:retrospettiva}
\section{Sprint 1}
\begin{description}
    \item [Svolgimento e sviluppo] Sfruttando il plugin \texttt{sbt-github-actions} sono state generati file per la configurazione della continuous integration.
    I membri del gruppo hanno approfondito la conoscenza del pattern ECS, alcune caratteristiche distintive di Scala 3.
    Si è configurato \texttt{scalafmt} per garantire uno stile uniforme nello sviluppo del progetto.
    Infine, è stato realizzato un design architetturale di massima.
    \item [Considerazioni finali] Lo sprint si è concluso nei tempi previsti e senza evidenziare particolari problematiche.
\end{description}
\section{Sprint 2}
\begin{description}
    \item[Svolgimento e sviluppo] Il secondo sprint è stato incentrato sulla realizzazione del design di dettaglio di World ed Entity.
    Sono stati implementati i principali metodi per l'aggiunta e rimozione dei componenti.
    È stata anche implementata la struttura dati \texttt{IterableMap}.
    \item[Considerazioni finali] Uno dei product backlog item (As a framework user I want to define World's views) non è stato concluso
    in tempo e pertanto è stato rimandato allo sprint successivo.
    I rimanenti product backlog item sono stati conclusi nei tempi prestabiliti.
\end{description}
\section{Sprint 3}
\begin{description}
    \item[Svolgimento e sviluppo] Durante lo sprint è stata completata l'implementazione delle \texttt{View} e realizzata una prima versione del DSL.
    Poiché grazie all'implementazione delle \texttt{View} il cuore della libreria può considerarsi concluso, sono stati realizzati dei benchmark
    per verificare il rispetto del requisito non funzionale~\ref{itm:nf1} ottenendo discreti risultati preliminari.
    \item[Considerazioni finali] Al termine dello sprint è emerso un problema implementativo che ha comportato un aumento significativo nello sforzo
    stimato per completare due dei backlog item (As a framework user I want to define and use Systems e As a framework user I want to update the World's state);
    perciò è stato spostato allo sprint successivo.
\end{description}
\section{Sprint 4}
\begin{description}
    \item[Svolgimento e sviluppo] Il team si è focalizzato sulla realizzazione dei \texttt{System} i quali hanno richiesto uno sforzo condiviso e collaborazione fra i vari membri. In questo modo
    è stato possibile risolvere le problematiche emerse senza ulteriori ritardi.
    Allo stesso tempo si è proseguito con la realizzazione del DSL; si prevede di ultimarlo nello sprint successivo.
    \item[Considerazioni finali] Lo sprint si è concluso senza problemi da evidenziare.
\end{description}