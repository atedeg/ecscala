\chapter{Processo di sviluppo}\label{ch:processo-di-sviluppo}
\section{Metodologia di lavoro}\label{sec:metodologia-di-lavoro}
Il processo di sviluppo adottato dal team si basa sulla versione semplificata di \textit{SCRUM} consigliata dal docente:
sono presenti le figure di \textit{Product Owner} e di \textit{Domain expert}, vengono svolti sprint a cadenza settimanale e si sono redatti
\textit{Product Backlog} e \textit{Sprint Backlog}.
Il team ha applicato il più possibile la tecnica di sviluppo \texttt{test-driven} e largamente sfruttato la modalità collaborativa di \textit{pair programming} che ha
permesso di ottimizzare e rendere più veloce l'implementazione del codice, nonché di individuare rapidamente eventuali bug.

\subsection{Organizzazione degli Sprint}\label{subsec:organizzazione-sprint}
Il primo sprint è stato organizzativo: sono stati identificati gli elementi di base del processo di svilppo, è stata preparata la build del progetto ed è stata prodotta la prima versione del \textit{Product backlog}
secondo i requisiti identificati per il progetto.
È stato inoltre necessario concordare il significato di ``done'' per capire quando uno sprint item potesse considerarsi concluso;
ne è emerso che ciò avviene se e solo se il codice che ne realizza le funzionalità soddisfa i seguenti punti:
\begin{itemize}
    \item i test vengono eseguiti con successo
    \item il codice è ben formattato
    \item è presente la \textit{Scaladoc}
    \item la coverage viene eventualmente decrementata al più del 5\%
    \item il codice prodotto viene controllato e approvato dagli altri membri del gruppo
\end{itemize}
Per quanto riguarda gli sprint successivi, ad inizio settimana il team si è incontrato di persona per la fase di \textit{Sprint Planning} durante la quale, a partire dagli item del \textit{Product Backlog}, è stato prodotto
lo \textit{Sprint Backlog} e sono stati assegnati gli item ai membri.
Al termine di ogni sprint è stata scritta la retrospettiva per verificare l'attuale stato di avanzamento del progetto ed, eventualmente,
posporre gli sprint item non conclusi alla settimana seguente.
Inoltre, ogni membro ha opportunamente aggiornato lo \textit{Sprint Backlog} con l'effort rimanente al completamento del proprio item.

\section{Strumenti utilizzati}\label{sec:strumenti-utilizzati}
A supporto del processo sopra descritto, il team ha principalmente adottato gli strumenti messi a disposizione da \textit{GitHub}: sono state utilizzate le \textit{Issue} per
tenere traccia degli item degli \textit{Sprint Backlog} e assegnarli ai singoli membri;
inoltre, si è ricorso a \textit{Pull Request} da feature branch per permettere la \textit{code review}.
Per avere una visione globale e chiara dell'andamento dello sviluppo, sono stati tracciati e tenuti in versione nel repository del progetto - su file .csv - tutti gli \textit{Sprint Backlog} e il \textit{Product Backlog}.

È stato verificato in modo automatico che il codice prodotto rispettasse i requisiti prestabiliti nella \textit{definition of done} tramite le \textit{GitHub Actions} configurate opportunamente.

In ultimo, si è utilizzato il \textit{Semantic Versioning} per le release.
