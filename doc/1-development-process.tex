\chapter{Processo di sviluppo}\label{ch:processo-di-sviluppo}
\section{Metodologia di lavoro}\label{sec:metodologia-di-lavoro}
Il processo di sviluppo adottato dal team si basa sulla versione semplificata di \textit{SCRUM} consigliata dal docente:
sono presenti le figure di \textit{Product Owner} e di \textit{Domain expert}, vengono svolti sprint a cadenza settimanale e sono stati redatti
\textit{Product Backlog} e \textit{Sprint Backlog}.
Inoltre, è stata applicata il più possibile la tecnica di sviluppo Test-driven.
Il team ha, inoltre, largamento sfruttato la tecnica di \textit{Pair programming} adottando lo sviluppo collaborativo in diverse occasioni.
Questo ha permesso di ottimizzare e rendere più veloce l'implementazione del codice, nonchè di individuare rapidamente eventuali bug.

\subsection{Organizzazione degli Sprint}\label{subsec:organizzazione-sprint}
Il primo sprint è stato organizzativo durante il quale stati identificati gli elementi di base del processo di svilppo, preparata la build del progetto e prodotta la prima versione del \textit{Product backlog}
secondo i requirements identificati per il progetto.
Abbiamo inoltre stabilito la nostra definizione di "done":
uno sprint item può dirsi concluso e non viene inserito nello sprint successivo se e solo se il codice che ne realizza le funzionalità soddisfa i seguenti punti:
\begin{itemize}
    \item i test vengono eseguiti con successo
    \item il codice è ben formattato
    \item è presente la \textit{Scaladoc}
    \item la coverage viene eventualmente decrementata al più del 5\%
    \item il codice prodotto viene controllato e approvato dagli altri membri del gruppo
\end{itemize}
Per quanto riguarda gli sprint successivi, ad inizio settimana il team si è incontrato di persona per la fase di \textit{Sprint Planning} durante la quale, a partire dagli item del \textit{Product Backlog}, viene prodotto
lo \textit{Sprint Backlog} e vengono assegnati gli item ai membri.
Al termine di ogni sprint è stata scritta la retrospettiva per verificare l'attuale stato di avanzamento del progetto ed, eventualmente,
postporre gli sprint item non conclusi alla settimana seguente.
Inoltre, ogni membro ha opportunamente aggiornato lo \textit{Sprint Backlog} con l' \textit{effort} rimanente al completamento del proprio item.

\section{Strumenti utilizzati}\label{sec:strumenti-utilizzati}
A supporto del processo sopradescritto, il team ha principalmente adottato gli strumenti messi a disposizione da GitHub, utilizzando i meccanismi di \textit{Issue} per
tenere traccia e assegnare ai singoli membri gli item degli \textit{Sprint Backlog}, e di \textit{Pull Request} da feature branch per permettere la \textit{code review}.

Inoltre, per avere una visione globale e chiara dell'andamento dello sviluppo, sono stati tracciati e tenuti in versione nel repository del progetto - su file .csv - tutti gli \textit{Sprint Backlog} e il \textit{Product Backlog}.

Per verificare in modo automatico che il codice prodotto rispettasse i requisiti prestabiliti nella \textit{definition of done}, sono state utilizzate le GitHub Actions configuarate opportunamente.

In ultimo, è stato utilizzato il \textit{Semantic Versioning} per le release.