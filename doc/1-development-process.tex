\chapter{Processo di sviluppo}\label{ch:processo-di-sviluppo}
\section{Metodologia di lavoro}\label{sec:metodologia-di-lavoro}
Il processo di sviluppo adottato dal team si basa sulla versione semplificata di \textit{SCRUM} consigliata dal docente:
sono presenti le figure di \textit{product owner} e di \textit{domain expert}, sono stati svolti sprint a cadenza
settimanale e si sono redatti \textit{product backlog} e \textit{sprint backlog}.
Il team ha applicato il più possibile la tecnica di sviluppo \textit{test-driven} e largamente sfruttato la modalità
collaborativa di \textit{pair programming} che ha permesso di ottimizzare e rendere più veloce l'implementazione del
codice, nonché d'individuare rapidamente eventuali bug.

\subsection{Organizzazione degli Sprint}\label{subsec:organizzazione-sprint}
Il primo sprint è stato organizzativo: sono stati identificati gli elementi di base del processo di sviluppo, è stata
preparata la build del progetto ed è stata prodotta la prima versione del product backlog secondo i requisiti
identificati per il progetto.
È stato inoltre necessario concordare la \textit{definition of done} per capire quando uno sprint item potesse
considerarsi concluso;
ne è emerso che ciò avviene solo se il codice che ne realizza le funzionalità soddisfa i seguenti punti:
\begin{itemize}
    \item i test vengono eseguiti con successo
    \item il codice è ben formattato
    \item è presente la Scaladoc
    \item la coverage viene, eventualmente, decrementata al più del 5\%
    \item il codice prodotto viene controllato e approvato dagli altri membri del gruppo
\end{itemize}
Per quanto riguarda gli sprint successivi, a inizio settimana il team si è incontrato di persona per la fase di
\textit{sprint planning} durante la quale, a partire dagli item del product backlog, è stato prodotto lo
sprint backlog e sono stati assegnati gli item ai membri.
Al termine di ogni sprint è stata scritta la retrospettiva per verificare l'attuale stato di avanzamento del progetto
ed, eventualmente, posporre alla settimana seguente gli sprint item non conclusi.
Inoltre, ogni membro ha opportunamente aggiornato lo sprint backlog con l'effort rimanente al completamento del
proprio item.

\section{Strumenti utilizzati}\label{sec:strumenti-utilizzati}
A supporto del processo sopra descritto, il team ha principalmente adottato gli strumenti messi a disposizione da
GitHub: sono state utilizzate le issue per tenere traccia degli item degli sprint backlog e assegnarli ai singoli
membri;
inoltre, si è ricorso a pull request da \textit{feature branch} per permettere la code review.
Per avere una visione globale e chiara dell'andamento dello sviluppo, sono stati tracciati e tenuti in versione nel
repository del progetto - su file .csv - tutti gli sprint backlog e il product backlog.

È stato verificato in modo automatico che il codice prodotto rispettasse i requisiti prestabiliti nella
definition of done tramite le \textit{GitHub Actions} configurate opportunamente.

Per le release su \textit{Sonatype} è stato utilizzato il plugin \texttt{sbt-ci-release}~\cite{sbt-ci-release}
opportunamente configurato per eseguire il rilascio direttamente dalla CI\@.

In ultimo, si è utilizzato il \textit{Semantic Versioning} per le release.
