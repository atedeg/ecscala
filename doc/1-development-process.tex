\chapter{Processo di sviluppo}\label{ch:processo-di-sviluppo}
\section{Metodologia di lavoro}\label{sec:metodologia-di-lavoro}
Il processo di sviluppo adottato dal team si basa sulla versione semplificata di \textit{SCRUM} consigliata dal docente:
sono presenti le figure di \textit{Product Owner} e di \textit{Domain expert}, vengono svolti sprint a cadenza settimanale e sono stati redatti
\textit{Product Backlog} e \textit{Sprint Backlog}.
Inoltre, è stata applicata il più possibile la tecnica di sviluppo Test-driven.
Il team ha largamento sfruttato la tecnica di \textit{Pair programming} adottando lo sviluppo collaborativo in diverse occasioni che ha permesso di velocizzare
l'implementazione del codice e il rilevamento di bug.

\subsection{Organizzazione degli Sprint}\label{subsec:organizzazione-sprint}
Il primo sprint realizzato è quello organizzativo nel quale sono stati identificati gli elementi di base del processo di svilppo, preparata la build del progetto e prodotta la prima versione del \textit{Product backlog}
secondo i requirement del framework identificati.
Inoltre, abbiamo definito la definition of done:


Per quanto riguarda gli sprint successivi, ad inizio settimana il team si è incontrato per la fase di \textit{Sprint Planning} durante la quale, a partire dagli item del \textit{Product Backlog}, viene prodotto
lo \textit{Sprint Backlog} e vengono assegnati gli item ai membri.
Al termine di ogni sprint è stata scritta la retrospettiva per verificare l'attuale stato di avanzamento del progetto ed, eventualmente,
postporre gli sprint item non conclusi alla settimana seguente.
Inoltre, ogni membro ha opportunamente aggiornato lo \textit{Sprint Backlog} con l' \textit{effort} rimanente al completamento del proprio item.

\subsection{Definition of done}\label{subsec:definition-of-done}
Uno sprint item può dirsi concluso e non viene inserito nello sprint successivo se e solo se il codice che ne realizza le funzionalità soddisfa i seguenti punti:
\begin{itemize}
    \item i test vengono eseguiti con successo
    \item è presente la \textit{Scaladoc}
    \item il codice prodotto viene controllato e approvato dagli altri membri del gruppo
    \item la coverage viene eventualmente decrementata al più del 5\%
\end{itemize}

\section{Strumenti utilizzati}\label{sec:strumenti-utilizzati}
A meno di eventuali bug fixes, ad ogni sprint item corrisponde una \textit{Pull Request}

- Descrizione di come abbiamo utilizzato github issue
- Utilizzo di file csv per tenere traccia degli sprint backlog item e product backlog
- Intero processo con PR (CI che passa -> Formattazione, Test e coverage), approvazione da tutti i membri delle PR, pair-programming, test-driven, semantic versioning.
- Librerie che non ci sono per Scala3