\chapter{Retrospettiva}\label{ch:retrospettiva}
\section{Sprint 1}\label{sec:sprint-1}
\begin{description}
    \item [Svolgimento e sviluppo] Sfruttando il plugin \texttt{sbt-github-actions} sono state generati file per la configurazione della continuous integration.
    I membri del gruppo hanno approfondito la conoscenza del pattern ECS, alcune caratteristiche distintive di Scala 3.
    Si è configurato \texttt{scalafmt} per garantire uno stile uniforme nello sviluppo del progetto.
    Infine, è stato realizzato un design architetturale di massima.
    \item [Considerazioni finali] Lo sprint si è concluso nei tempi previsti e senza evidenziare particolari problematiche.
\end{description}
\section{Sprint 2}\label{sec:sprint-2}
\begin{description}
    \item[Svolgimento e sviluppo] Il secondo sprint è stato incentrato sulla realizzazione del design di dettaglio di World ed Entity.
    Sono stati implementati i principali metodi per l'aggiunta e rimozione dei componenti.
    È stata anche implementata la struttura dati \texttt{IterableMap}.
    \item[Considerazioni finali] Uno dei product backlog item "As a framework user I want to define World's views" non è stato concluso
    in tempo e pertanto è stato rimandato allo sprint successivo.
    I rimanenti product backlog item sono stati conclusi nei tempi prestabiliti.
\end{description}
\section{Sprint 3}\label{sec:sprint-3}
\begin{description}
    \item[Svolgimento e sviluppo] Durante lo sprint è stata completata l'implementazione delle \texttt{View} e realizzata una prima versione del DSL\@.
    Poiché grazie all'implementazione delle \texttt{View} il cuore della libreria può considerarsi concluso, sono stati realizzati dei benchmark
    per verificare il rispetto del requisito non funzionale~\ref{itm:nf1} ottenendo discreti risultati preliminari.
    \item[Considerazioni finali] Al termine dello sprint è emerso un problema implementativo che ha comportato un aumento significativo nello sforzo
    stimato per completare due dei backlog item "As a framework user I want to define and use Systems" e "As a framework user I want to update the World's state";
    perciò è stato spostato allo sprint successivo.
\end{description}
\section{Sprint 4}\label{sec:sprint-4}
\begin{description}
    \item[Svolgimento e sviluppo] Il team si è focalizzato sulla realizzazione dei \texttt{System} i quali hanno richiesto uno sforzo condiviso e collaborazione fra i vari membri.
    In questo modo
    è stato possibile risolvere le problematiche emerse senza ulteriori ritardi.
    Allo stesso tempo si è proseguito con la realizzazione del DSL\@.
    \item[Considerazioni finali] Lo sprint si è concluso senza problemi da evidenziare.
\end{description}
\section{Sprint 5}\label{sec:sprint-5}
\begin{description}
    \item[Svolgimento e sviluppo] Non avendo fin'ora riscontrato ritardi significativi nello sviluppo, il team ha deciso di aggiungere meno item allo sprint per poter conciliare impegni accademici.
    Sono stati implementati \texttt{ExcludingView} ed \texttt{ExcludingSystem} ed è stato arricchito il DSL per permettere l'aggiunta di \texttt{System} e la rimozione di \texttt{Component}.
    \item[Considerazioni finali] Lo sprint si è concluso evidenziando incosistenze nella sintassi del DSL che saranno corrette nel prossimo sprint.
    Inoltre, intendiamo fare refactoring dell'interfaccia di \texttt{World} che secondo noi ha margini di miglioramento ma essendo una modifica consistente non poteva essere completata in questo sprint.
\end{description}
\section{Sprint 6}\label{sec:sprint-6}
\begin{description}
    \item[Svolgimento e sviluppo] Lo sprint si è focalizzato principalmente sulla definizione dell'architettura per la demo.
    Inoltre, sono state apportate migliorie alla sintassi del DSL ed è stata semplificata l'interfaccia del \texttt{World}.
    Infine, è stato effettuato il primo rilascio della libreria dal momento che sono state implementate tutte le funzionalità minime necessarie.
    \item[Considerazioni finali] Abbiamo deciso di spostare uno degli sprint item "Refactor World's interface" a quello successivo, in quanto ritenuto di bassa priorità.
\end{description}
\section{Sprint 7}\label{sec:sprint-7}
\begin{description}
    \item[Svolgimento e sviluppo] Lo sprint ha avuto come obiettivo principale la realizzazione della demo.
    Inoltre è stato effettuato il refactoring della classe \texttt{World} e le corrispondenti parti nel DSL\@.
    \item[Considerazioni finali] Sono stati spostati allo sprint successivo gli item "Refactor World's interface" e "As a framework user I would higly appreciate a handy DSL to perform these operations".
\end{description}
\section{Sprint 8}\label{sec:sprint-8}
\begin{description}
    \item[Svolgimento e sviluppo] In questo sprint il team ha ultimato la demo e il DSL. In aggiunta, ha rifattorizzato l'implementazione di alcune interfacce del framework.
    \item[Considerazioni finali] Il progetto può dirsi concluso.
\end{description}