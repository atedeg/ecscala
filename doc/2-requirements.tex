\chapter{Requisiti}\label{ch:requisiti}
Per l'individuazione dei requisiti è stato innanzitutto analizzato il pattern ECS e la terminologia adottata.
In Tabella~\ref{tab:glossario} è riportato un glossario con i principali concetti del pattern:
\begin{table}[H]
    \begin{tabular}{p{0.17\linewidth}p{0.76\linewidth}}
        \toprule
        \textbf{World}     & Contiene più \Entity e i rispettivi \Component.
        Permette la registrazione di più \System che utilizza per aggiornare lo stato dei \Component \\
        \textbf{View}      & Rappresenta il sottoinsieme delle \Entity di un \World con i soli \Component specificati \\
        \textbf{Entity}    & Oggetto definito dai rispettivi \Component\\
        \textbf{Component} & Rappresenta una particolare caratteristica da modellare per una \Entity\\
        \textbf{System}    & Aggiorna i \Component di tutte le \Entity di una determinata \View secondo una logica 
        definita dall'utilizzatore\\
        \bottomrule
    \end{tabular}\caption{\label{tab:glossario}Glossario dei termini del dominio.}
\end{table}

\section{Business}\label{sec:business}
L'obiettivo è quello di realizzare un framework che permetta di applicare in maniera semplice il pattern ECS\@.
I requisiti di business individuati sono:
\begin{enumerate}[label=\textbf{\ref{sec:business}.\arabic*}]
    \item \label{itm:b1} Dev'essere possibile utilizzare in maniera semplice ed efficiente il pattern ECS
    \item \label{itm:b2} Il framework dev'essere sufficientemente flessibile da poter realizzare simulazioni e videogiochi.
    In particolare dev' essere possibile:
    \begin{enumerate}[label=\textbf{\ref{itm:b2}.\arabic*}]
        \item \label{itm:bb3} Realizzare una simulazione del moto di palle da biliardo in un tavolo da gioco
    \end{enumerate}
\end{enumerate}

\section{Utente}\label{sec:utente}
I requisiti utente sono sviluppati considerando il punto di vista dello sviluppatore che dovrà utilizzare il framework.
In particolare:
\begin{enumerate}[label=\textbf{\ref{sec:utente}.\arabic*}]
    \item \label{itm:u1} Dev possibile creare il \World che contiene tutte le \Entity
    \item \label{itm:u2} Dev'essere possibile creare e rimuovere \Component
    \item \label{itm:u3} Dev'essere possibile creare e rimuovere \Entity
    \item \label{itm:u4} Dev'essere possibile creare \System
    \item \label{itm:u5} Dev'essere possibile utilizzare un DSL per effettuare le operazioni sopra elencate
    \item \label{itm:u6} Per l'utente è importante avere un esempio di utilizzo del framework
\end{enumerate}

\section{Funzionali}\label{sec:funzionali}
I requisiti funzionali, ricavati da quelli utente, sono:
\begin{enumerate}[label=\textbf{\ref{sec:funzionali}.\arabic*}]
    \item \label{itm:f1} Definire uno o più \World
    \begin{enumerate}[label=\textbf{\ref{itm:f1}.\arabic*}]
        \item \label{itm:ff2} Far avanzare lo stato del \World, comportando l'aggiornamento delle sue \Entity
    \end{enumerate}
    \item \label{itm:f2} Definire \Component
    \item \label{itm:f3} Creare \Entity all'interno di un \World
    \item \label{itm:f4} Rimuovere \Entity dal \World in cui si trovano
    \item \label{itm:f5} Manipolare lo stato delle \Entity
    \begin{enumerate}[label=\textbf{\ref{itm:f5}.\arabic*}]
        \item \label{itm:ff3} Aggiungere \Component alle \Entity
        \item \label{itm:ff4} Rimuovere \Component dalle \Entity
    \end{enumerate}
    \item \label{itm:f6} Creare diverse tipologie di \View che selezionino alcune \Entity del \World
    \begin{enumerate}[label=\textbf{\ref{itm:f6}.\arabic*}]
        \item \label{itm:ff5} \View per ottenere tutte le \Entity del \World che possiedono i \Component specificati
        \item \label{itm:ff6} \View per ottenere tutte le \Entity del \World che non possiedono nessuno dei \Component
        specificati
    \end{enumerate}
    \item \label{itm:f7} Creare diverse tipologie di \System
    \begin{enumerate}[label=\textbf{\ref{itm:f7}.\arabic*}]
        \item \label{itm:ff7} \System per manipolare gli specifici \Component delle \Entity
        \item \label{itm:ff10} \System che non operano sui \Component delle \Entity ma permettono di eseguire operazioni
        a ogni aggiornamento dello stato del \World
        \item \label{itm:ff11} \System che consentono di escludere alcune \Entity sulla base dei \Component specificati
    \end{enumerate}
    \item \label{itm:f8} Registrare \System nel \World
    \item \label{itm:f9} Fornire un DSL per
    \begin{enumerate}[label=\textbf{\ref{itm:f9}.\arabic*}]
        \item \label{itm:ff12} Definire \System
        \item \label{itm:ff13} Manipolare \Entity
        \item \label{itm:ff14} Manipolare lo stato del \World
    \end{enumerate}
\end{enumerate}


\section{Non funzionali}\label{sec:non-funzionali}
Considerando gli scenari d'uso elencati al punto~\ref{itm:b2}, il \System deve rispettare il seguente requisito:
\begin{enumerate}[label=\textbf{\ref{sec:non-funzionali}.\arabic*}]
    \item \label{itm:nf1} Aggiornare velocità e posizione di $10\;000$ \Entity in non più di 10ms
\end{enumerate}


\section{Implementativi}\label{sec:implementativi}
Per la realizzazione del framework verrano utilizzati:
\begin{enumerate}[label=\textbf{\ref{sec:implementativi}.\arabic*}]
    \item \label{itm:i1} Scala 3 come linguaggio principale
    \item \label{itm:i2} ScalaTest~\cite{scalatest} per la realizzazione di unit test
\end{enumerate}
