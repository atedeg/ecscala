\chapter*{Introduzione}
\addcontentsline{toc}{chapter}{Introduzione}
Il progetto ha come obiettivo la realizzazione di un framework che consenta l'implementazione del pattern
architetturale \ECS (Entity Component System).
Questo è un pattern tipicamente usato nel settore videoludico, che favorisce i principi di
\textit{composition over inheritance} e \textit{data-oriented design}.
Due esempi significativi di uso del pattern ECS sono i videogiochi \textit{Overwatch} (2016) e \textit{Minecraft}
(2011).
% TODO: Riguardare talk GDC ed espandere un po' sui vantaggi

I tre concetti chiave di ECS sono:
\begin{itemize}
    \item \Entity: un oggetto distinto che esiste nella simulazione, che non ha dati né comportamenti,
    \eg una palla da biliardo, un nemico di un videogioco, un proiettile, eccetera.
    \item \Component: un elemento che descrive una particolare caratteristica di una \Entity senza descriverne
    l'effettivo comportamento, \eg posizione, velocità, salute, informazioni per il rendering, eccetera.
    \item \System: una funzione che viene eseguita su ogni \Entity che possiede tutti i \Component richiesti e ne
    legge o modifica i loro valori;
    tramite questa si dà l'effettivo comportamento alle \Entity.
    Alcuni esempi di \System possono essere un sistema di rendering che richiede i componenti posizione e immagine,
    un sistema di calcolo dei danni che richiede il componente salute, eccetera.
\end{itemize}

Esistono diverse librerie open source che permettono l'implementazione del pattern \ECS; fra le più importanti risulta
EnTT, libreria scritta in \CC\ usata dal videogioco \textit{Minecraft}.
Nell'ambiente JVM troviamo invece \textit{Artemis} e \textit{Ashley}, entrambe scritte in Java, che però non possono
vantare un uso in applicazioni di rilievo.

Il nostro obiettivo è quindi quello di creare una libreria che faccia della type safety il suo maggior punto di forza,
alla quale si aggiunge un comodo DSL per scrivere codice idiomatico e compatto.
Sebbene le prestazioni siano normalmente d'importanza spesso critica, abbiamo deciso di non realizzare tutte le
ottimizzazioni implementate dalle altre liberie, in quanto esulano dall'obiettivo del corso.