\chapter{Implementazione}\label{ch:implementazione}
\section{Scala 3}\label{sec:scala-3}

\subsection{Impliciti}\label{subsec:impliciti}

\subsection{Macro}\label{subsec:macro}

\subsection{Context function}\label{subsec:context-function}

\subsection{Match types}\label{subsec:match-types}

\subsection{Type class}\label{subsec:type-class}

\subsection{Extension method}\label{subsec:extension-method}

\subsection{Problemi riscontrati}\label{subsec:problemi-riscontrati}

\section{Benchmarks}\label{sec:benchmarks}

\section{Suddivisione del lavoro}\label{sec:suddivisione-del-lavoro}

\subsection{Di Domenico}\label{subsec:nicolò-di-domenico}

Il mio ruolo nel progetto è stato principalmente quello d'implementare le \texttt{View} e i \texttt{System}, nonché
applicare ove necessario eventuali ottimizzazioni per rispettare il requisito non funzionale \ref{itm:nf1}.

\subsubsection{View}

Essendo le \texttt{View} ed \texttt{ExcludingView} progettate come indicato in Figura~TODO, si è rivelato utile costruire un
iteratore astratto che permettesse di fattorizzare tutto il codice comune.
In particolare si noti come la \texttt{ExcludingView} itera su un sottoinsieme delle entità ritornare dalla stessa
\texttt{View}, che invece non esprime vincoli di esclusione di componenti.
Per questo motivo è stato creato il \texttt{BaseViewIterator}, che si occupa di recuperare dal
\texttt{ComponentsContainer} tutte le mappe che associano ciascuna entità all'istanza (se presente) del componente di
ciascun tipo.
Infine, partendo da tali mappe, si occupa di testare l'appartenenza alla view di ciascuna entità e, in caso affermativo,
estrarne tutti i componenti richiesti.

Una semplice ottimizzazione utilizzata per rendere più veloce l'iterazione di una \texttt{View} consiste nell'individuare
la mappa di componenti con meno elementi;
una volta trovata, si iterano tutte le entità contenute al suo interno e per ognuna si controlla che sia presente in tutte
le altre mappe.
In caso affermativo, si procede all'estrazione di tutti i suoi componenti richiesti.

\subsubsection{System}

Un \texttt{System} è un blocco di codice che può essere aggiunto al \texttt{World} e viene chiamato a ogni aggiornamento
di stato della simulazione.
Ogni \texttt{System} ha due metodi:
\begin{itemize}
    \item \texttt{shouldRun}: metodo booleano che indica se il \texttt{System} dev'essere eseguito
    \item \texttt{update}: metodo che contiene il codice arbitrario da eseguire
\end{itemize}

Su di esso sono stati costruiti \texttt{IteratingSystem} ed \texttt{ExcludingSystem}, due wrapper type-safe
rispettivamente per \texttt{View} ed \texttt{ExcludingView}.
Questi due sistemi definiscono i seguenti metodi:
\begin{itemize}
    \item \texttt{shouldRun}: metodo booleano che indica se l'\texttt{IteratingSystem} dev'essere eseguito
    \item \texttt{before}: metodo eseguito prima di iterare su tutte le entità con i componenti richiesti
    \item \texttt{update}: metodo eseguito per ciascuna entità contenente le operazioni da effettuare sui suoi
    componenti richiesti
    \item \texttt{after}: metodo eseguito dopo aver iterato su tutte le entità con i componenti richiesti
\end{itemize}

Il metodo \texttt{update} deve ritornare una \texttt{CList} dei nuovi componenti;
grazie all'utilizzo delle \texttt{CList} (descritte in dettaglio in TODO) verrà verificato a tempo di compilazione che i
nuovi componenti abbiano lo stesso tipo di quelli su cui opera il sistema.
Per permettere la cancellazione di alcuni dei componenti, è stato implementato un \textit{match type} chiamato
\texttt{Deletable[L~<:~CList]}, come mostrato nel Listato~\ref{lst:deletable};
il suo scopo è quello di permettere l'inserimento in ogni posizione della \texttt{CList} un componente speciale chiamato
\texttt{Deleted}: in tal caso il componente in quella posizione sarà cancellato.

\lstinputlisting[label={lst:deletable}, caption=Implementazione del tipo \texttt{Deletable[L~<:~CList]}]{./code/deletable.scala}

L'\texttt{ExcludingSystem} è completamente analogo all'\texttt{IteratingSystem}, in quanto cambia solo la \texttt{View}
usata per ottenere le entità sulle quali iterare.